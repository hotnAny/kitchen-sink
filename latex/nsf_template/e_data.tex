\section*{Data Management Plan}
% Proposals must include a supplementary document of no more than two pages labeled ``Data Management Plan". This
% supplementary document should describe how the proposal will conform to NSF policy on the
% dissemination and sharing of research results (see AAG Chapter VI.D.4)

We expect to produce the following kinds of artifacts: 
(1) software,
(2) data and analysis results, 
(3) publications, and 
(4) educational materials.

\subsection*{Software Sharing Plan}
Our project team believes strongly in open-source software development and supporting open data access as far as is feasible given ethical, privacy, security, and intellectual property concerns.

The team has a great track record of developing open-source software that is used across a range of application domains, ranging from \xx
Recent examples include \xx
All software that we develop for the current proposal will also be made publicly available.

Specifically, this research will produce \xx

Our tools for end users will use simple, easy-to-use, and learnable interfaces and abstractions, as well as cleanly designed APIs to make them broadly
accessible.
We will also publish user-friendly documentation and offer tutorials that will complement the technical descriptions that we will publish in academic papers. 
We will establish a publicly-available project web page to serve as a common portal for all publicly-released data stemming from this work, along with access to the infrastructure itself.  The project web page will be easily discoverable from the PIs' home page.  No fees will be charged for access to data or software. 
The PIs will make sure the software is freely available to science and engineering researchers and educators in the non-profit sector. 
The PIs will also support other individuals or teams to develop new research tools or systems by using or extending our software.

Implementations of the new tools developed in this proposed work will be added to the project website as they are ready for use.  
If applicable, we will also distribute them via other distribution channels---e.g., Python's package installer \texttt{pip} or the Chrome Web Store.
We have already used both \texttt{pip} and the Chrome Web Store successfully for existing projects.
All open-source software, tutorials, papers, data, and other information will be hosted on the project website, backed up by hosting services such as \texttt{github.com}.

\subsection*{Data and Analysis Results}
Our team will perform human subject studies of our prototype under the oversight of an Institutional Review Board.  
Data collected in these studies may include details of user interactions---mouse clicks and keyboard interactions---and survey or interview responses. 
These will be used to understand how participants use the prototype. 
We may also make screen recordings of participants interacting with the prototype. 
We may collect email addresses in order to compensate participants.
Email addresses will be the only identifiers collected. 
There will be no link between the email addresses and the anonymous tool usage data. 
All data will be secured on password-protected computers. 
Student investigators will have access to the data. 
We will destroy the list of participant email addresses within a prescribed time period after each study session. 
We will not destroy anonymous system usage data.
Retained datasets will not contain identifiable information.
We will make retained datasets available to the research community via \url{github.com}.
Likewise, we will make the materials we develop for running these human subjects studies available to the research community via \url{github.com}.

The team has access to servers that can ensure the long-term preservation of the public data and open-source software, and we do not envision any obstacle to making this data accessible to all. 
All the PIs will be jointly responsible for managing and disseminating data and software generated by the project. 
For the public data and open-source software, since it will be broadly available, we do not anticipate any problems with the dissemination if any of the PIs were to leave their current institution.

\subsection*{Publications}
We plan to publish our findings in top venues and conferences, including technical reports, articles in conferences, journals, and workshops, describing significant findings from work conducted under this grant.

\subsection*{Educational materials}
We plan to disseminate our findings in the research community by giving workshops and tutorials as well as writing survey articles.  
We also plan to incorporate findings in graduate-level courses. 
All material resulting from this activity will be made available online, including lecture notes, slides, and homework exercises. 

As described in the Software Sharing Plan, we will also develop educational materials specifically for teaching the use of our prototype, with the aim of supporting users from related community.  
Additionally, we will make all resultant educational materials available via our project webpage.  
Materials may include tutorials, documentation, and lectures.