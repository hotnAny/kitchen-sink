\section*{Facilities, Equipments, \& Other Resources}
% This section of the proposal is used to assess the adequacy of the resources available to perform the effort proposed to satisfy both the Intellectual Merit and Broader Impacts review criteria. Proposers should describe only those resources that are directly applicable. Proposers should include an aggregated description of the internal and external resources (both physical and personnel) that the organization and its collaborators will provide to the project, should it be funded. Such information must be provided in this section, in lieu of other parts of the proposal (e.g., budget justification, project description). The description should be narrative in nature and must not include any quantifiable financial information. Reviewers will evaluate the information during the merit review process and the cognizant NSF Program Officer will review it for programmatic and technical sufficiency.

% last update: 3/2024
\subsubsection*{PI Chen's Lab at UCLA}
The UCLA Human-Computer Interaction Laboratory (HCIL) maintains access to ample lab space (about 1000 sq ft) for the PI and graduate students for the expected duration of the project.  There will be multiple Windows and Linux workstations available for student use for various computing needs (\eg developing AI models and system prototypes).  Laboratory space for the proposed infrastructure will also be provided by the department for the extended lifespan of the equipment, beyond the duration of this proposal.  It is equipped with the necessary power and ethernet necessary for the operation of the various equipment as well as for use by researchers working in the space. There is access to multiple quite rooms available for conducting human subject studies, such as usability testing and interview.
HCIL also has access to the support from the Henry Samueli School of Engineering and Applied Science at UCLA on education and research activities through a variety of shared computing resources for both research and teaching activities. The School provides networking, disk storage and backup, general-purpose timesharing access to multi-user systems, email and workstation support. The School's central facility is for the support of research, and is open to undergraduate and graduate students, faculty and staff. The Department network is connected to the UCLA campus backbone via a gigabit connection. An 802.11n wireless network is available throughout the Department. The Department network is also linked to the School's network that includes IBM servers and workstations and PC-based labs, supporting classes and giving access to undergraduate students. 