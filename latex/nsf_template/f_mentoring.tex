\section*{Mentoring Plan}

The PI’s mentoring plan aims to equip graduate students with skills and experiences essential for excelling in diverse HCI career paths. This plan supports the holistic development of students, enhancing both technical expertise and broader professional competencies.

\subsubsection*{Career Counseling/Advising} In addition to weekly meetings, the PI will conduct quarterly one-on-one career counseling sessions, focusing on students' career goals, recent opportunities, and notable success stories. Through these discussions, the PI will help students set and evaluate their goals, introduce them to key professionals in their field, and provide guidance on networking, internships, job talks, and interviews. Students will also be encouraged to attend workshops and conferences to broaden their professional network and disseminate their work.

\subsubsection*{Training in Proposal, Publication, and Presentation Preparation} Students will receive detailed guidance on writing research articles, presenting at conferences, and preparing grant proposals. The PI will provide feedback on drafts and encourage publication in high-impact venues. Students will also improve communication skills by presenting their work in group meetings, workshops, and conferences, with support on slide design, poster creation, and narrative clarity. Senior students may receive mentorship on proposal writing (focused on PhD fellowships and scholarships), with the PI reviewing drafts, discussing funding options, and guiding the structuring of strong proposals.

\subsubsection*{Improvement of Teaching and Mentoring Skills} The PI will support students in developing teaching and mentoring skills. Interested students will be encouraged to participate in teaching assistantships, with the PI providing constructive feedback. The PI will share effective pedagogy practices and encourage senior students to mentor undergraduates or newer lab members, fostering a collaborative environment. Resources on building strong mentor-mentee relationships will also be provided.

\subsubsection*{Collaboration with Researchers from Diverse Backgrounds} The PI will promote interdisciplinary collaboration, guiding students to communicate effectively across disciplines and leverage external expertise to enhance their research. Opportunities to collaborate within and outside the institution will be actively supported, enriching students’ research perspectives.

\subsubsection*{Training in Responsible Professional Practices} Ethics and responsibility are central to this plan. The PI will provide regular guidance on research integrity, data management, and human subjects' ethical considerations. Topics like authorship, conflicts of interest, and peer review will be discussed during individual meetings to instill professional responsibility.

In supporting these areas, the PI’s goal is to develop well-rounded HCI professionals capable of excelling in varied roles as a scholar, educator, and research community member and making impactful contributions in and beyond academia.