\section*{Summary of Proposed Work}


This request is for an REU supplement to NSF Award \awardno, entitled \awardtitle.  
This grant, which ends on \awardenddate, 
\awardsummary.
\xac{update the macros of meta info in main.tex}

We seek this small amount of additional funding to provide an excellent research experience for undergraduates while developing new human-AI collaborative techniques for physicians.  
This REU supplement plays a critical role in our Broadening Participation in Computing program which uses undergraduate research experiences such as the ones proposed here to increase the percentage of undergraduate women who persist to graduate school.  
More generally, these research experiences are an important tool for recruitment of future graduate students into HCI research at the PI's institute.

\subsection*{Description of Research and Benefit to Student}
This REU request would support \numstudents undergraduate researchers during the summer to participate in the design, development, and study of interactive systems
\xx ... \xac{add detailed descirption here}

Previous REU support of this project contributed to the development of novel human-AI collaborative techniques published at 
\xx ... \xac{(top HCI venue) with (student) as a co-author}

This REU will benefit the undergraduate researchers by developing their research skills and their understanding of human-centered design methods. 
The upcoming summer project is extremely well-suited for undergraduates because the scope is well-defined, the approach has been grounded in sufficient preliminary work, yet important questions remain to be answered.
Furthermore, the project provides an ideal introduction for an undergraduate to rich intersections of HCI + AI.  
The expectation is that a conference paper (most likely CHI whose deadline is often in September) will result from this research.

\subsection*{PI's Prior Experience Involving Undergraduates in Research}
The PI has over a decade of experience mentoring undergraduate students in research and HCI Research Group has a reputation for providing excellent opportunities for undergraduates to participate in HCI research.
Other then regular REU programs, the PI also frequently hosted independent study courses (\eg two students' project resulted in a demo at IUI 2019) and the \emph{Learning by Research} program that provides opportunities for non-computing students (\eg in Psychology and CogSci) to participate in quarter-long research projects (\eg two recent students contributed to and co-authored an IEEE ICHI paper).

\subsection*{Description of Mentoring Provided for each REU student}

This REU experience is part of an overall undergraduate retention activity in the UCLA Electrical \& Computer Engineering Department to increase the participation of undergraduates in our research programs.   

The REU students will be fully integrated in the research life of the PI's lab. 
Each new undergraduate researcher is paired with a graduate student mentor and has the opportunity to participate in the weekly meeting with the PI.

Additionally, the UCLA Samueli School of Engineering provides a Summer Undergraduate Research Program (SURP) that supports undergraduate researchers new to research with training in the basic skills of research including how to present technical material to a variety of audiences, how to prepare a research abstract, how to read and discuss journal papers, and how to prepare and present an effective research poster.
The SURP culminates with a research symposium in which all participants will present posters on their research to faculty, corporate representatives, and fellow students. 
The students supported under this grant will participate in this symposium and present their research.

\subsection*{Relationship of the REU Funding to the Original Award}

This request is for an REU supplement allows additional researchers to contribute to the design, development, and study of techniques proposed in the original award, thus bringing in positive impact on making further overall progress.

\subsection*{Process and Criteria for Selecting Students}

The undergraduate researchers will be selected from a pool of applications including the new Fast Track Cohort (The ECE retention program for top applicants). 
REU positions are widely advertised on the departmental website and through campus-wide research portal.  
This year we will also expand recruitment to undergraduate research fairs sponsored by the UCLA School of Engineering and engagement with student clubs (\eg Bruin AI, ACM, and IEEE).
As soon as our REU awards are confirmed we will select our summer 2024 participants.

The PI's team will interview both Fast Track students as well as others who have applied for the positions.  
Students are selected based on their interest in the project, their academic ability, prior experience, as well as social, communication, and computer/software skills. 
In addition to gender balance, every effort is made to recruit minority students.
We will make every effort to consider students from a diverse background, especially those that are traditionally underrepresented in the computing community and/or have limited resources to participate in computing education and research.

All REU students supported by this funding will be US citizens or Permanent residents.

\subsection*{Specifics About the REU Request}
As described in the Budget Justification section, this is a request for \numstudents undergraduate researchers to receive a stipend of \$7000 each and the registration fee of \$500 each to participate in the UCLA Samueli Summer Undergraduate Research Program.  
The stipend and registration fee is for a 10-week program.