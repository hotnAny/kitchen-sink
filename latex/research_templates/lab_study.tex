% 
% - the tone is future tense for proposal purposes

% 
% start with a DESCRIPTION of why conducting this study
We will conduct a \xx study to understand/validate/\xx \xx
% 
% formalize into specific RESEARCH QUESTIONS that are 1) properly scoped with 2) clear expected types of answers
Specifically, we seek to investigate the following questions:
\begin{itemize} [leftmargin=0.25in]
    \item [RQ1] \xx
    \item [RQ2] \xx
\end{itemize}

% HYPOTHESES: based on some theory, prior findings, or rationale, you could reasonably expect the results of your study to be ...
For RQ1, we hypothesize that \xx

% 
\subsection{Participants}
We will recruit
\xx % sample size
participants,
\xx male, \xx female, % gender
aged \xx to \xx, % age
from \xx % source, such as a local university
via \xx % recruiting approach

The inclusion criteria are \xx; % how to select participants
there are no exclusion criteria. % what kinds of participants should be excluded

%
Our study has been approved by the IRB of our institute.
% optional: the detailed consent process (if some reviewers ask for it)
% We first went through the informed consent process with each participant, including an explanation of the kinds of data that would be collected, i.e., audio and screen recordings and their usage log of interacting with our tool. Specifically, the log consisted of mouse/keyboard input events and user-generated data (e.g., their answers to reflective questions), which would be stored in the browser's cookies and deleted after we processed the raw data after each study session.

% describe WHAT participants do without mentioning HOW, which is for the experimental design subsection
\subsection{Tasks \& Procedure}
We started each participant with \xx % introduction, tutorial, or practices

We then asked each participant to \xx % do the actual tasks
% 
% a detailed walkthrough of how a task starts, unfolds, and finishes.
% 
We instructed participants to \xx as quickly and accurately as possible. % optional

Finally, we conducted an interview to gather participants' feedback on \xx

Each study session lasted for about \xx, and each participant was compensated with \$\xx for their time.

% 
\subsection{Design}
We employed % a repeated measures within/between-subject factorial design
% 
The independent variables were \xx (\xx, \xx, and \xx) % describe the variable and its different levels
We counterbalanced the presentation of \xx across participants.
% optional (from Vogel & Baudisch's Shift): The 6 Target Sizes were paired with each of the 4 Directions and presented in random order within each block. The experiment had 1 practice block and 3 timed blocks for each Technique and Contact combination.
% 
The dependent variables were \xx
% 
% optional
% In sum, the experimental design was:
% # independent variable (aa, bb, ...) $\times$\\
% ...

% 
\subsection{Apparatus}
We conducted the experiment on \xx % what device
that has % relevant specs, e.g., screen size
We implemented \xx % the software
using \xx
% optional: describe any client-server setup if the computing and UI run on separate places

% 
\subsection{Measurement \& Analysis}
% external recording
We recorded the screen and audio of the whole process.
We analyzed this data by \xx
% 
% internal logging
Integrated with \xx, we logged participants' system usage data, \eg \xx
We analyzed this data by \xx
% 
During the interview, we asked participants \xx
Our analysis followed the thematic approach [\xx]: the first author transcribed participants' responses to formulate the initial codes, which were then reviewed by two other authors. Then, disagreements were resolved via discussion between the authors.