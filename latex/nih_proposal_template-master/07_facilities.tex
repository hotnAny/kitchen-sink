{\noindent \bf \Large FACILITIES \& OTHER RESOURCES}\\

{\bf The PI's lab resources}. 
The UCLA Human-Computer Interaction Research Laboratory (HCIR) maintains access to ample lab space (about 1000 sq ft) for the PI and graduate students for the expected duration of the project. 
There will be multiple Windows and Linux workstations available for student use for training data-driven AI models or hosting servers for distributed interactive systems. 
The Laboratory space is equipped with the necessary power and ethernet ports necessary for the operation of the various equipment as well as for use by researchers working in the space.
The Laboratory maintain access to various sensors related to interactive systems (EMG, IMU, RGB-D, capacitive touch, ultrasonic, UWB), actuators for robotic systems (step motor, solenoid, quadcopters), and basic fabrication machines for rapid prototyping (laser cutter, paper cutter, 3D printer).
There is a dedicated conference room that provides private and quiet space with established computers and recording devices for conducting human subject laboratory studies.

{\bf UCLA ECE department and engineering school resources}.
Laboratory space for the proposed infrastructure will also be provided by UCLA ECE Department for the extended lifespan of the equipment beyond the duration of this proposal. 
HCIR is also located close by and has access to the recently-established Maker Space provides extended use of laser cutters, 3D printers, electronics workstations (e.g., soldering, PCB) available for creating research prototypes.
The UCLA Engineering School provides networking, disk storage and backup, general-purpose timesharing access to multiuser systems, email and workstation support. The School's central facility is for the support of research, and is open to undergraduate and graduate students, faculty and staff. The Department network is connected to the UCLA campus backbone via a gigabit connection. An 802.11n wireless network is available throughout the Department. The Department network is also linked to the School's network that includes IBM servers and workstations and PC-based labs, supporting classes and giving access to undergraduate students.

{\bf Resources for data collection}.
% Although this project is not focused on innovating AI models per se, it still requires training AI upon which the proposed CDSS can be built.
Although the proposed P-GEM is generalized to any medical data, the scope of this project will focus on histological imaging data to enable in-depth investigation.
% For radiologists we primarily use Chest X-ray (CXR) images as they are the most commonly used and easily acquirable. 
% Typically hundreds of CXR images are required to train an AI model, which we will obtain by collaborating with the Department of Radiology at UCLA where such data has been routinely and systematically archived for research purposes.
% For histological data, w
We will primarily use brain tissue slides as they represent a type of condition amenable for using generating synthetic data that simulates their various patterns.
Typically tens of patients' slides (up to 10 slides per patient) are needed to train a generative model, which we will obtain by working with neuropathologists at the Medical School, UCLA Health, and University of Kansas Medical Center (where a previous UCLA collaborator has moved).
The UCLA Department of Pathology and Laboratory Medicine includes full facilities and equipment for
light microscopic, ultrastructural, immunohistochemical, and molecular study of human tissues. The
Department is fully computerized with modern HP PCs, all linked to the network through UCLA Pathnet,
as well as slide scanners for whole slide imaging. The Section of Neuropathology which evaluates several
hundred brain tumors each year is housed in a facility of approximately 1700 sq. ft. in the former Brain
Research Institute.

{\bf Resources for recruiting physicians for studies}.
Physicians in UCLA Health are highly supportive of cross-disciplinary collaboration and will collaborate with the PI as experts for user research (\eg observing how radiologists work in their reading rooms) and participants in a usability evaluations of the proof-of-principle system prototypes. 