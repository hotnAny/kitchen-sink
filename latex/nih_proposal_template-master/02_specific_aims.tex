\section*{Specific Aims}
\vspace{-1em}

% ----------------------------------------------------
% 1. INTRODUCTORY PARAGRAPH

% <!-- 1.1 opening sentence -->
% The opening sentence must be an interest-grabbing sentence that immediately establishes the relevance of the proposal to human health.
% • You want to convey that, by supporting your proposal, the reviewers will be helping NIH accomplish its goals.

% <!-- 1.2 current knowledge -->
% Help less expert members of the panel get up to speed with respect to what is known about the topic of the application.

% <!-- 1.3 gap in knowledge -->
% Introduce what is missing in sentences that identify the gap in knowledge. It is the gap in knowledge that is holding back the field and is what you will address in the application.

{\bf We propose to develop and release P-GEM---an open-source software tool that enables pathologists to interactively oversee and control the generation of synthetic histological data}.
Synthetic data generated by algorithms (\eg Generative Adversarial Networks, or GANs) promises to lower the cost of data collection and to compensate for the scarcity of data in uncommon diseases.
However, at present, such promises are hindered by a lack of pathologists ``in the loop''---most generative algorithms are based on highly abstract and complex neural networks that function as ``black boxes'', inaccessible to pathologists who are non-experts in computation.
As a result, due to the inability to control the generative process, pathologists are unlikely to accept the use of synthetic data in downstream tasks (\eg for disease research or training data-driven AI models).

% ----------------------------------------------------
% 2. WHAT WHY WHO PARAGRAPH

% <!-- 2.1 long-term goal -->
% The long term goal projects the continuum of research that you will pursue over the course of multiple periods of grant support.
% • This component tells the reviewers what the “big picture” of your research program is.

% <!-- 2.2 objective of this application -->
% Defines what it seeks to accomplish.
Our {\bf Main Objective} is to develop a tool that bridges the gulf of expertise and empower pathologists to oversee and control algorithms that generate synthetic histological data.
We call this tool P-GEM---{\underline P}athologist-in-the-Loop \underline{GE}neration of {\underline M}edical data.
% 
Developing P-GEM is {\bf High-Risk} because generative algorithms are notoriously difficult to control even for professional computer programmers---how can pathologists who have much less computational training manage the complexity of such algorithms without having to spend unreasonable amount of extra effort?
When successful, realizing P-GEM is {\bf High-Reward} because it will democratize and accelerate the adoption of synthetic medical data in clinical settings, thus translating the rapid development in generative algorithms to the betterment of medicine and public health.
% <!-- 2.2 the central hypothesis -->
% Must link to the objective, because the objective will be accomplished by testing the central hypothesis.
% 
% <!-- 2.4 rationale -->
% Conveys why you want to conduct the proposed research. Your rationale should tell the reviewers what will become possible after the research is conducted that is not possible now!
% 
{\bf Clinical significance}: Leveraging our ongoing collaboration with two medical centers (UCLA and KUMC), we will closely collaborate with the \xx departments and aim at enabling \xx pathologists to use P-GEM for generating synthetic histological data with \xx pattern---an important indicator of the uncommon \xx [disease].

% ----------------------------------------------------
% 3. SPECIFIC AIMS PARAGRAPH
{\bf P-GEM operate in two modes for controlling algorithms to generate histological data following a pathologist-defined direction}. The extrinsic mode (Aim 1): pathologists can use their familiar medical concepts to `poke' the black-box algorithm and `steer' its generative direction, \eg to produce data with more or less of certain patterns; The intrinsic mode (Aim 2): pathologists can break into the black-box and take a `guided tour' into the algorithm's vast space of generated examples to tease out medically-meaningful generative directions at their disposal.

% <!-- 3.1 Aim 1 -->
{\bf Aim 1: the extrinsic mode---enabling physicians to control the generative process using extrinsically-defined medical concepts}.
To achieve this mode, we address two research questions:

% \vspace{-1em}
% \begin{itemize}[leftmargin=0.25in]
    % \item 
    \underline{Aim 1a. How to enable pathologists to define medical concepts with as little effort as possible?} 
    We represent a concept as two sets of contrastive examples (\eg one with and the other without the \xx pattern).
    In contrast to existing approaches that require costly localizing and annotating many small patches of such patterns, we will build on our prior work on multiple instance learning, whereby pathologists only need to coarsely specify large areas in the low-power field containing a target pattern. Further, to improve the quality of the concept representation, we will recommend to pathologists `near-miss' examples that are highly similar to but are {\it not} the target pattern.

    \underline{Aim 1b. How to influence a generative algorithm using such pathologist-defined medical concepts?}
    Once a concept is defined, it can be used to `steer' a generative algorithm at various layers of its neural networks. We will develop feedforward visualization to help a pathologist understand the effect of steering early \vs middle \vs late layers. 
    We will conduct an experiment to compare such concept-enabled control with a baseline of simply browsing a large collection of generated data points.
% \end{itemize}


% <!-- 3.2 Aim 2 -->
{\bf Aim 2: the intrinsic mode---enabling physicians to see, visualize, and explore intrinsically-learned concepts in the generative space}.
We have developed a tool in a non-medical domain for general users to interactively discover editing directions of a GAN model, which we will further develop and integrate into P-GEM to work with histological data.
Specifically, pathologists will discover generative directions by iteratively performing a `scatter/gather' operation: first gathering clusters of data (provided by P-GEM) that share common patterns and then scattering these clusters to further separate them by their differences.
We will conduct experiments to evaluate whether pathologists can utilize this mode to discover a concept that already exists in GAN's latent space for controlling the generative algorithm.

% ----------------------------------------------------
% 4. PAYOFF PARAGRAPH

% <!-- 4.1 expected outcomes -->
% Articulate the expected products of the research. This paragraph details the payoff that the reviewers

% <!-- 4.2 positive impact -->
% Summarize the general impact of the expected outcomes. The positive impact statement should make clear that, collectively, the outcomes will advance your field vertically, as well as contribute to the mission of the NIH Institute/Center that you are targeting

% tool