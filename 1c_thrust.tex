% QUALITATIVE STUDY
% - partially based on the Creswell book
% - applies to: interviews, contextual inquiry, and other types of formative studies

% [intro]
% 1. establishing the problem leading to the study
% 2. casting the problem within the larger scholarly literature
% 3. discussing deficiencies in the literature about the problem, and
%   3a. Write about areas overlooked by past studies, including topics, special statistical treatments, and significant implications
%   3b. Discuss how the present study addresses these deficiencies and provides a unique contribution to the literature.
% 4. targeting an audience and noting the significance of the problem for this audience.

% [purpose statement]
% The purpose of this study is (was/will be) to ________ (understand/describe/develop/discover) the ________ (central concept being studied, optionally followed by a brie definition) for  ________ (the unit of analysis: a person/processes/groups/site) using a ________ (method of inquiry, see above) resulting in a ________ (type of finding).

% [research questions]
% 1. the grand tour question (a statement of the question being examined in the study in its most general form, … posed as a general issue so as not to limit the inquiry)
% 2. a small, limited number of subquestions.
% tips:
% - for grounded-theory type of study, a research question starts with ‘what’ or ‘how’ and seeks to ‘discover’ something
% - Avoid using words with a directional orientation, e.g., affect, influence, impact, determine, cause, and relate. For HCI, some of these words might be fine, e.g., ‘affect’ as long as it is not leading. 

% [hypotheses]

% [participants]

% [protocol / process]
% - how to structure the study process into multiple parts?

% [analysis & expected outcome]